\documentclass[12pt]{article}
\usepackage[all, stdclass]{lix}
\usepackage{graphicx}
\usepackage{svg}
\usepackage{circuitikz}
\usepackage{amsmath}
\usepackage{bookmark}
\svgsetup{
  inkscapepath=assets/,  % Path to the directory containing your SVG files
  svgpath=assets/        % Path to the directory containing your SVG files
}
\usepackage{float}
\usepackage{hyperref}
\usepackage{times}
\usepackage{graphicx}


%----------EDIT COVER INFO HERE -----------------%

\def \LOGOPATH {assets/birzeit-logo.png}
\def \DEPARTEMENT {Department of Electrical \& Computer Engineering}
\def \COURSENUM {ENEE2103}
\def \COURSENAME {Circuits and Electronics Laboratory}
\def \REPORTTITLE {Filters}
\def \STUDENTNAME {Mohammad Abu-Shelbaia}
\def \PARTNERAN {Nidal Zabade}
\def \PARTNERBN {Mahmoud Shihab}
\def \PARTNERAID {1200153}
\def \PARTNERBID {1180000}
\def \STUDENTID {1200198}
\def \INSTRUCTOR {Dr. Mahran Quran }
\def \ASSISTANT {Eng. Raffah Rahal}
\def \REPORTNUM {5}

%--------------------BORDERS----------------------------%
% \usepackage{parskip}
% \setlength{\parskip}{0pt}
% \geometry{top=1.54cm}
% \usepackage{everyshi}
% \usepackage{tikz}
% \EveryShipout{%
%     \begin{tikzpicture}[overlay,remember picture]
%         \draw [line width=0.5pt]
%             ($ (current page.north west) + (1cm,-1cm) $)
%             rectangle
%             ($ (current page.south east) + (-1cm,1cm) $);
%     \end{tikzpicture}
% }
%------------------------------------------------%


\begin{document}

\begin{titlepage}
    \vfill
    \begin{center}
        \includegraphics[width=0.7\textwidth]{\LOGOPATH} \\
        \hfill \\
        \Large{\DEPARTEMENT} \\
        \Large{\COURSENUM\;-\;\COURSENAME} \\
        \vfill
        \textbf{\LARGE{Experiment \#\REPORTNUM}} \\
        \textbf{\LARGE{\REPORTTITLE}}
    \end{center}
    \vfill
    \begin{flushleft}
        \Large{\textbf{Prepared by:}\\ \STUDENTNAME\quad\STUDENTID} \\
        \Large{\textbf{Partners:}\\ 
        \begin{tabular}{@{}l@{\quad}l}
            \PARTNERAN & \PARTNERAID \\
            \PARTNERBN & \PARTNERBID \\
        \end{tabular}} \\
        \Large{\textbf{Instructor:} \INSTRUCTOR} \\
        \Large{\textbf{Assistant:} \ASSISTANT} \\
        \Large{\textbf{Section:} 4}\\
        \LARGE{\textbf{ }}\\
        \LARGE{\textbf{ }}\\
        \LARGE{\textbf{ }}\\
        \Large{\textbf{Date:} \today}\\
    \end{flushleft}
    \vfill
\end{titlepage}

\clearpage
% --------------- ABSTRACT ------------------%
\begin{abstract}
    \noindent
    \textbf{Abstract:} \\
    \textbf{Keywords:} \\
\end{abstract}
\clearpage
%--------------- TABLES --------------------------------%
\tableofcontents
\clearpage
\setlength{\parskip}{\baselineskip}%
\listoffigures
\clearpage
\listoftables
\clearpage
\pagenumbering{arabic}
%-------------- CONTENT ---------------------%
\h{Theory}
Filters in circuits allow ciruin frequincies to pass through while blocking undesired frequencies. Filters are classified into two types: passive and active filters. Passive filters are made of passive components such as resistors, capacitors, and inductors. Active filters are made of active components such as transistors and op-amps. A filter order is the number of reactive components in the filter.
Furthermore, filters have four basic types: low pass, high pass, band pass, and band stop. 
\hh{Passive Filters}
\hhh{Low Pass Filter}
As the name suggests, this type of filters allow low frequencies to pass through while blocking high frequencies.
\begin{figure}[H]
    \centering
    \resizebox{0.40\textwidth}{!}{%
    \begin{circuitikz}
    \tikzstyle{every node}=[font=\LARGE]
    % \draw (-13,13) to[short, -*] (-13,13);
    \draw (-5,7) to[R] (-2.25,7);
    \draw (-2.25,7) to[C] (-2.25,4.75);
    \draw[] (-2.25,4.75) to[short] (-5,4.75);
    \draw [](-2.5,4.75) to[short] (-1,4.75);
    \draw [](-2.25,7) to[short] (-1,7);
    \draw [](-5,7) to[short, -o] (-5.25,7);
    \draw [](-5,4.75) to[short, -o] (-5.25,4.75);
    \draw [](-1,4.75) to[short, -o] (-0.75,4.75);
    \draw [](-1.25,7) to[short, -o] (-0.75,7);
    \node [font=\large] at (-5.25,5.875) {Vi};
    \node [font=\large] at (-0.5,5.875) {Vo};
    \end{circuitikz}
    }
    \resizebox{0.40\textwidth}{!}{%
    \begin{circuitikz}
    \tikzstyle{every node}=[font=\LARGE]
    % \draw (-13,13) to[short, -*] (-13,13);
    \draw (-5,7) to[L] (-2.25,7);
    \draw (-2.25,7) to[R] (-2.25,4.75);
    \draw[] (-2.25,4.75) to[short] (-5,4.75);
    \draw [](-2.5,4.75) to[short] (-1,4.75);
    \draw [](-2.25,7) to[short] (-1,7);
    \draw [](-5,7) to[short, -o] (-5.25,7);
    \draw [](-5,4.75) to[short, -o] (-5.25,4.75);
    \draw [](-1,4.75) to[short, -o] (-0.75,4.75);
    \draw [](-1.25,7) to[short, -o] (-0.75,7);
    \node [font=\large] at (-5.25,5.875) {Vi};
    \node [font=\large] at (-0.5,5.875) {Vo};
    \end{circuitikz}
    }
    \label{fig:Low-Pass}
    \caption{$1^{st}$ Order Passive Low-Pass filters}
\end{figure}
% \begin{equation} \label{eq1}
\begin{equation}
\begin{aligned}
            V_o &= \frac{X_c}{X_c + X_r} \times V_i  \quad\quad &V_o &= \frac{X_r}{X_r + X_l} \times V_i\\
            X_c &= \frac{1}{j\omega c}  &X_l &= j\omega l
\end{aligned}
\end{equation}
% \end{equation}
The freqeuncy is inversely proportinal to the impedance $X_c$, and $X_c$ is directly proportinal to $V_o$, and by the same logic the frequency is directly proportinal to $X_l$, and $X_l$ is inversely proportinal to $V_o$, which indicates that higher freqeuncies generate low voltage (reject), and low freqeuncies generate high voltage (pass).\\ \\
The cut-off frequency is the frequency at which point of inversion occurs, which indeicates what a high and low frequeinces are,and its given by:
\begin{equation}
    \begin{aligned}
        f_c = \frac{1}{2\pi R C} \quad\quad or \quad\quad f_c = \frac{R}{2\pi L}
    \end{aligned}
\end{equation}
\hhh{High Pass Filter}
This type of filter is a complement of the low-pass filter, since it rejects low frequeinces  and passes high ones.
\begin{figure}[H]
    \centering
    \resizebox{0.40\textwidth}{!}{%
    \begin{circuitikz}
    \tikzstyle{every node}=[font=\LARGE]
    % \draw (-13,13) to[short, -*] (-13,13);
    \draw (-5,7) to[C] (-2.25,7);
    \draw (-2.25,7) to[R] (-2.25,4.75);
    \draw[] (-2.25,4.75) to[short] (-5,4.75);
    \draw [](-2.5,4.75) to[short] (-1,4.75);
    \draw [](-2.25,7) to[short] (-1,7);
    \draw [](-5,7) to[short, -o] (-5.25,7);
    \draw [](-5,4.75) to[short, -o] (-5.25,4.75);
    \draw [](-1,4.75) to[short, -o] (-0.75,4.75);
    \draw [](-1.25,7) to[short, -o] (-0.75,7);
    \node [font=\large] at (-5.25,5.875) {Vi};
    \node [font=\large] at (-0.5,5.875) {Vo};
    \end{circuitikz}
    }
    \resizebox{0.40\textwidth}{!}{%
    \begin{circuitikz}
    \tikzstyle{every node}=[font=\LARGE]
    % \draw (-13,13) to[short, -*] (-13,13);
    \draw (-5,7) to[R] (-2.25,7);
    \draw (-2.25,7) to[L] (-2.25,4.75);
    \draw[] (-2.25,4.75) to[short] (-5,4.75);
    \draw [](-2.5,4.75) to[short] (-1,4.75);
    \draw [](-2.25,7) to[short] (-1,7);
    \draw [](-5,7) to[short, -o] (-5.25,7);
    \draw [](-5,4.75) to[short, -o] (-5.25,4.75);
    \draw [](-1,4.75) to[short, -o] (-0.75,4.75);
    \draw [](-1.25,7) to[short, -o] (-0.75,7);
    \node [font=\large] at (-5.25,5.875) {Vi};
    \node [font=\large] at (-0.5,5.875) {Vo};
    \end{circuitikz}
    }
    \label{fig:High-Pass}
    \caption{$1^{st}$ Order Passive High-Pass filters}
\end{figure}
\begin{equation}
    \begin{aligned}
                V_o &= \frac{X_r}{X_c + X_r} \times V_i  \quad\quad &V_o &= \frac{X_l}{X_r + X_l} \times V_i\\
                X_c &= \frac{1}{j\omega c}  &X_l &= j\omega l
    \end{aligned}
    \end{equation}
According to the above equations, the frequency is inversely proportinal to the impedance $X_c$, and $X_c$ is inversely proportinal to $V_o$, and by the same logic the frequency is directly proportinal to $X_l$, and $X_l$ is directly proportinal to $V_o$, which indicates that higher freqeuncies generate high voltage (pass), and low freqeuncies generate low voltage (reject).\\ \\
The cut-off frequency is the frequency at which point of inversion occurs, which indeicates what a high and low frequeinces are,and its given by:
\begin{equation}
    \begin{aligned}
        f_c = \frac{1}{2\pi R C} \quad\quad or \quad\quad f_c = \frac{R}{2\pi L}
    \end{aligned}
\end{equation}
\hhh{Band Pass Filter}
\hhh{Band Stop Filter}
\hh{Active Filters}
\h{Procedure and Data Analysis}
\h{Discussion}
\h{Conclusion}
\h{References}
\h{Appendix}
\end{document}





